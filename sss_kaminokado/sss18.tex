%tex file for ISCIE International Symposium on 
%            Stochastic Systems Theory and Its Applications
%
%

%latex209
%\documentstyle[sss,epsfig]{article}

%latex2e
\documentclass[a4paper]{article}
\usepackage{latexsym}
\usepackage{sss}
\usepackage{graphicsx} % for pdf, bitmapped graphics files
\usepackage{epsfig} % for postscript graphics files
\usepackage{amsmath}

\begin{document}
\date{}
\title{\LARGE{\bf
The 50th ISCIE International \\
 Its Applications\\
Instruction for Authors}
}
\author{
Shingo Kaminokado \\
Dept. of Electrical and Electronic Systems, Okayama University of Science\\
Ridai-cho 1-1, Kita-ku, Okayama, 700-0005 Japan\\
\\E-mail: \texttt{xxx@sci-sss.org}
}

\maketitle
\thispagestyle{empty}
%ABSTRACT
\abstract{
    We have developed a weeding robot(Aigamo Robot). This robot works automatically using 
    Global Navigation Satellite System (GNSS). Previous independent positioning measure 
    included several meter differences and dispersion of the positions and directions. 
    Therefore, in this study, we implement the Real Time Kinematic-GNSS (RTK-GNSS) on 
    the right and left side of the robot in order to reduce these errors. T
    his method can obtain the position and direction more acculately when the robot is not 
    only in operation, but also under suspending. Moreover, this method can suppress 
    the errors within about 10 cm to the south-southeast when the path tracking.
}


\section{Introduction}
我々の研究グループは小型の水田除草用ロボット(アイガモロボット)を開発している.[1]
このロボットは水田内を自動で走行し,除草を行う.
ロボットは広大な環境を走行するため,ロボットの自己位置を知ることは重要である.
ロボットの自己位置を知るには,カメラやビーコンなどを用いる手法が存在する.
しかしながら,天候などの外乱の影響を受けやすいため,我々のロボットには適用が困難である.
そのため,自己位置を知る手法としてGlobal Navigation Satellite System(GNSS)を採用する.
従来のロボットは単独測位法を用いて,ロボットの位置を取得している.
しかし,位置や向きの誤差やばらつきが大きくずれる問題がある.
そのため,より高精度なReal Time Kinematic-GNSS(RTK-GNSS)を用いる.
GNSSモジュールは主に2種類の搬送波を受信する.2周波受信機は高価で大きいため,我々の小型のロボットには搭載することができない.
一方で,近年マルチGNSS技術によって衛生数が増加し,1周波の搬送波のみ取得できる安価で小型なGNSSモジュールが市場に出回るようになった.
そこで我々のグループは安価で小型なGNSSモジュールを採用した.
しかし,一つのモジュールだけではロボットの向きの推定が困難である.我々のグループは2つのGNSSをロボットの左右に設置した.
先行研究ではそれらの精度を調査し,高精度な測位が可能であることを示した[2].
本研究では我々の測位システムに先行研究で取得したデータ適用したカルマンフィルタを導入することによって,
より高精度なGNSS測位システムの提案を行う.
カルマンフィルタの検証にはシミュレーションを用いる.


\section{カルマンフィルタの説明}
この章では提案するカルマンフィルタのモデルを説明をする.
本研究ではアイガモロボットを対象としており,このロボットは2次元平面上を移動する.カルマンフィルタの予測フェーズ
\begin{align}
    \hat{x}_{t+ \Delta t|t} &= F_{t} \hat{x}_{t} + B_{t} u_{t} \nonumber \\
                            &= \hat{x}_{t} + B_{t} u_{t}
    \label{eq:1}
\end{align}
%
%
\begin{align}
    \hat{P}_{t+ \Delta t|t} &= F_{t} P_{t|t} F_{t}^{T} + Q \nonumber \\
                            &= P_{t|t} + Q
    \label{eq:2}
\end{align}
%
%
\begin{equation}
    \begin{bmatrix}
    \hat{x}_{t+ \Delta t|t} \\
    \hat{y}_{t+ \Delta t|t} \\
    \hat{\theta}_{t+ \Delta t|t}
    \end{bmatrix} 
    =
    \begin{bmatrix}
        \hat{x}_{t|t} \\
        \hat{y}_{t|t} \\
        \hat{\theta}_{t|t}
    \end{bmatrix} 
    +
    \begin{bmatrix}
        \Delta t cos\hat{\theta}_{t|t} &0 \\
        \Delta t sin\hat{\theta}_{t|t} &0 \\
        0                              &{\Delta}t
    \end{bmatrix}
    \begin{bmatrix}
        v_{t} \\
        \omega_{t}
    \end{bmatrix}
    \label{eq:3} 
\end{equation}
%
%
\begin{equation}
    e_{t} = \hat{x}_{t} - x_{t} =
    \begin{bmatrix}
        \hat{x}_{t} - x_{t} \\
        \hat{y}_{t} - y_{t} \\
        \hat{\theta}_{t} - \theta_{t}
    \end{bmatrix}
    \label{eq:4}
\end{equation}
%
%

\begin{equation}
    P_{t|t} = E
    \begin{pmatrix}
        e_{t} e_{t}^{T}
    \end{pmatrix}
    \label{eq:5}
\end{equation}
%
%
\begin{equation}
    \begin{split}
    \begin{bmatrix}
    \sigma_{xx,t|t+\Delta t} &\sigma_{xy,t|t+\Delta t} &\sigma_{xz,t|t+\Delta t} \\
    \sigma_{yx,t|t+\Delta t} &\sigma_{yy,t|t+\Delta t} &\sigma_{yz,t|t+\Delta t} \\
    \sigma_{zx,t|t+\Delta t} &\sigma_{zy,t|t+\Delta t} &\sigma_{zz,t|t+\Delta t} 
    \end{bmatrix} \\
    = 
    \begin{bmatrix}
        \sigma_{xx,t|t} &\sigma_{xy,t|t} &\sigma_{xz,t|t} \\
        \sigma_{yx,t|t} &\sigma_{yy,t|t} &\sigma_{yz,t|t} \\
        \sigma_{zx,t|t} &\sigma_{zy,t|t} &\sigma_{zz,t|t}
    \end{bmatrix} 
    +
    \begin{bmatrix}
        q_{x}^{2} &0         &0 \\
        0         &q_{y}^{2} &0 \\
        0         &0         &q_{\theta}^{2}
    \end{bmatrix}
\end{split}
    \label{eq:6} 
\end{equation}
%
%
 カルマンフィルタの更新フェーズ
\begin{equation}
    y_{t} = z_{t} - H_{t}\hat{x}_{t+\Delta t|t}
    \label{eq:7}
\end{equation}
%
%
\begin{equation}
    S_{t} = R + H_{t}P_{t+\Delta t|t}H_{t}^{T}
    \label{eq:8}
\end{equation}
%
%
\begin{equation}
    \hat{x}_{t+\Delta t|t+\Delta t} = x_{t+\Delta t|t+\Delta t}
    \label{eq:9}
\end{equation}


\section{カルマンフィルタに用いる分散の取得実験}

Regarding the styles of your manuscript, please conform to the following 
%instructions. 

\subsection{Title}
The title should be centered across the top of the first page and should 
be in a distinctive point size or font.

\subsection{Author's Names and Address}
The authors' names and addresses should be centered below the title.
It is desirable that these lines are typed in at least eleven point font size,
but the particular point sizes and fonts are not critical and are left to the
direction of the authors.
Times new Roman 12 point is suggested.
Please include your E-Mail address. 

\subsection{Headings}
Main headings are to be column centered in a bold font without an underline.
They may be numbered, if so desired. 

Subheadings should be in a bold font or underlined lowercase with initial
capitals.
They should start at the left-hand margin on a separate line.

Sub-subheadings are to be in a bold font or underlined type.
They should be indented and run in at the beginning of the paragraph.

\subsection{Figures and Tables} 
Figures and photos should be consecutively numbered like Fig{.}~1, Fig{.}~2,
Figures should be inserted near their citation or at the end of the manuscript.
Large figures and tables may span across both columns if necessary.
Figure captions should be placed below the figures. 

\subsection{References}
List and number all references at the end of the paper as shown below.
Number reference citations consecutively in square brackets [1].

\subsection{Page Numbers}
Do not write page numbers on your manuscript.
These will be inserted later by the proceedings printer together with
the session number and conference identifications.


\section{Manuscript Submission}
Authors are requested to send their manuscripts electronically by
{\bf July 31, 2018} on the web:
\begin{center}
http://sci-sss.org/sss2018/sub/submission.php
\end{center}


\section{Conclusions}

Please make an extra effort to adhere to these guidelines as the quality
of the publications depends on you.
Thank you for your cooperation and contribution.
We are looking forward to seeing you at the 49th ISCIE International Symposium
on Stochastic Systems Theory and Its Applications (SSS'17).


\begin{thebibliography}{99}
\bibitem{aigamo}
M. Taku, O. Yoshiaki, O. Jun, N. Keita and N. Keitaro:
Mechanism of generating drawbar pull of rod wheel on loose soil,
{\it 22rd International Symposium on Artificial Life and Robotics.(AROB)}, Vol.22, No.4, 2017.
\bibitem{gnss}
H. J. Christopher and C. Eric:
Evolution of the global navigation satellitesystem (gnss),
{\it Proceedings of the IEEE}, Vol.96, No.12, pp.1902-1917, 2008.
\bibitem{rtk-gps}
K. Michio. N. Noboru, I. Kazunobu and T. Hideo: 
Field Mobile Robot navigated by RTK-GPS and FOG, 
{\it Journal of the Japanese Society of Agricultural Machinery}, Vol.63, No.5, pp.74-79, 2001.(in Japanese)

\bibitem{auto-weeding}
S. Masahiro, N. Yoshisada, T. Katsuhiko and K. Kyou: 

{\it Journal of the Japanese Society of Agricultural Machinery}, Vol.72, No.3, pp.276-282, 2010.(in Japanese)


\bibitem{camera-relate}
S.Shotaro, H. Zhencheng and F. Thomas:
Tracking ofFeatUre PointsforVisualSLAM with MultipleCameras
{\it The Institude of Image Information and Television Engineers}

\bibitem{reach rs} 
Emlid Reach RS, https://docs.emlid.com/eachrs/

\bibitem{reach} 
Emlid Reach, https://docs.emlid.com/each/

\end{thebibliography}
\end{document}
% end of sss10.tex
